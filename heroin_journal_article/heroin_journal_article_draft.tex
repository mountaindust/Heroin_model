\documentclass[12pt]{article}


\usepackage[numbers]{natbib}
\usepackage{graphicx} 
\usepackage{amssymb, amsmath, amsthm} 
\usepackage{fontenc} 
\usepackage{amscd,latexsym,amsfonts,amstext,amsbsy}
\usepackage{euscript} 
\usepackage{enumerate} 
\usepackage{color}  
\usepackage{physics}
\usepackage[latin1]{inputenc}
\usepackage{tikz}
\usepackage{mathrsfs}
\usetikzlibrary{shapes,arrows}
\usepackage{multicol}
\usepackage{comment}
\usepackage{color,soul}
\usepackage{combelow}
\bibliographystyle{apa}
\definecolor{applegreen}{rgb}{0.55, 0.71, 0.0}




\textwidth = 16 cm
\textheight = 24 cm
\oddsidemargin = 0.0 cm
\evensidemargin = 0.0 cm
\topmargin = -2 cm
\parskip = 0.2in
\parindent = 0.0in


\newtheorem{theorem}{Theorem}
\newtheorem{problem}[theorem]{Problem}
\newtheorem{exercise}[theorem]{Exercise}
\newtheorem{corollary}[theorem]{Corollary}
\newtheorem{lemma}[theorem]{Lemma}
\newtheorem{proposition}[theorem]{Proposition}
\newtheorem{proporties}[theorem]{Proporties}
\newtheorem{definition}[theorem]{Definition}
\newtheorem{definitions}[theorem]{Definitions}
\newtheorem{example}[theorem]{Example}
\newtheorem{remark}[theorem]{Remark}

 


\begin{document}
\textbf{\large{Modeling the Heroin Epidemic}} \\
Suzanne Lenhart, Tricia Phillips, Christopher Strickland \\

\textbf{Abstract} \\ \\
\textbf{Introduction} \\ \\
\textcolor{blue}{Unsure of which pieces to take out/are unnecessary, so I labeled the main ideas of the paragraphs.} \\ \\
\textcolor{green}{Heroin information} \\ \\
\textcolor{blue}{About the drug:} \\
Heroin is an illicit drug classified as an opioid and comes in the form of a white or brown powder or as a black substance resembling roofing tar. The drug is injected, sniffed, snorted or smoked and quickly enters the brain to bind to opioid receptors. It provides the user with feelings of euphoria, in addition to physical effects such as heavy feelings in the arms and legs, dry mouth, and sometimes nausea and vomiting \cite{NIH1, NIDA2}. There are short and long term negative effects on the body for using the drug, and consequently, it is currently considered a schedule I drug, meaning that there is no approved medical use of heroin and there is a high likelihood for abuse. \cite{DEA1, NIH1}. Addicted individuals have withdrawal symptoms such as restlessness, diarrhea, vomiting, and cold flashes, along with cravings for the drug which makes stopping use of the drop very difficult \cite{NIH1}. Options for treatment for heroin use involve medications such as buprenorphine, methadone, and naltrexone, in combination with counseling and behavioral therapies \cite{SAMSHA1, NIH1}. Heroin users build up a tolerance to the drug with repeated use and can overdose on the drug, in which their heart rate and breathing is slowed to a dangerous level without medical assistance \cite{NIDA2, NIH1}. Due to built-up tolerance, the risk of overdose is high when individuals stop their use of the drug for a period of time (i.e. while in recovery or hospitalized) and return to use. This is because they may return to the previous amount they were taking before, without knowing what their body can currently tolerate \cite{NIH2}.  

\textcolor{blue}{Statistics regarding heroin and why we should care:} \\
A national survey estimated that out of the United States national population of individuals 12 years of ago or older, 948,000 were heroin users in 2016, although this number may include under-reporting \cite{CDC2}. There were an estimated 13,219 heroin deaths in 2016, a more than six-fold increase from the year 2002, despite there being roughly half as many heroin users in 2002 \cite{NSDUH1}. This, in part, is due to the recent trend of lacing heroin with fentanyl, a surgical-grade opioid that is up to fifty times more potent than heroin alone, and therefore, users are unaware of the purity of the heroin they obtain \cite{CDC1, NIH2, Volkow2}. The effects of heroin, however, extend further than just the using individual. Due to the sharing of needles and other equipment involved in the injection of heroin, the human immunodeficiency virus (HIV) and the hepatitis C virus (HCV) are more easily contracted as transmittance occurs through bodily fluids. Moreover, risky sexual behavior is more likely under the influence, so for those whose method of use is via smoking or snorting, there is also an increased risk of transmitting and contracting these viruses. Women who use heroin during pregnancy put their babies at risk for neonatal abstinence syndrome in which the drug is passed along to the baby, resulting in dependency and thus, withdrawal symptoms upon birth \cite{NIDA2}. 

\textcolor{blue}{Heroin history:} \\
Heroin was first formulated in a hospital in 1874, but production on a commercial level did not begin until 1898, when the Bayer Company in Germany marketed and sold the new drug. Prescription heroin was deemed helpful for treating bronchitis, asthma and tuberculosis, but near the turn of the century, the dangers of the drug were starting to be revealed, such as it's potential for addiction. In 1920, the American Medical Association stated that heroin should not be manufactured, sold, or prescribed in the United States, and in combination with the amount of crime resulting from heroin use, a law was passed in 1924 prohibiting crude opium imports with the intention of heroin production. With the start of World War II, traffickers diluted the drug more and more, due to tighter border controls and fewer supplies entering the country \cite{UnitedNations}. \textcolor{red}{More heroin history: find reliable source for smuggling of heroin from China to US in 1930s; motor vehicle vs. drug overdose stats; ~750,000 heroin addicts from 1965-1970; purity level/price of heroin; provide argument for why the problem is so large now that we need data-driven models compared to just a theoretical heroin addiction model as done for decades.} 

\textcolor{green}{Fentanyl information} \\
\textcolor{red}{Put in background on fentanyl.} \\

\textcolor{green}{Prescription opioid information} \\
\textcolor{blue}{About prescription opioids and why we should care:} \\
Another large portion of opioids consist of prescription pain relievers which are either natural, such as morphine and codeine, semi-synthetic, such as oxycodone, hydrocodone, and oxymorphone, or synthetic, such as fentanyl and tramadol \cite{CDC3, TNMentalHealth2}. From 1991 to 2011, there was a near tripling of prescriptions that pharmacies distributed \cite{NIDA1}. This was in part due to a number of new opioids that were approved by the FDA for use, such as OxyContin, Actiq, Fentora, and Onsolis (fentanyl), in addition to other unapproved opioid products for pain management \cite{FDA1}. Moreover, in the early 2000s, drug manufacturers funded publications and physicians to support opioid use for pain control \cite{Mandell}. A national survey estimated there were 11.5 million individuals 12 years of age or older in the United States that experienced pain reliever misuse in the past year, referring to the year 2016 \cite{CDC2}. In this survey, misuse was defined as taking the prescription at a higher dose, more frequently or longer than prescribed, taking someone else's medication, or any other way not directed by a doctor \cite{SAMSHA3}. This subset of the opioid drug class is misused for a variety of reasons and the same survey asked individuals who reported prescription pain reliever misuse to give the reason and the source for their most recent misuse. The most prominent responses for reasons of misuse were to relieve physical pain, to feel good or get high, and to relax or relieve tension. The largest source was from friends/relatives or from a healthcare provider, followed by given a prescription or stolen from a health care provider \cite{CDC2}. Treatment options for opioid addiction are the same as heroin \cite{SAMSHA1}. In addition, pregnant women who take prescription opioids also put their babies at risk of neonatal abstinence syndrome, similar to heroin \cite{CDC5}. 

\textcolor{blue}{I left out the prescription opioid ``history" since was covered in opioid paper and not as much of our focus?}

\textcolor{green}{Connection between opioid and heroin abuse} \\ \\
\textcolor{blue}{Transitioning from prescription opioids to heroin:} \\
The misuse of prescription pain relievers leads some individuals to start heroin. According to the National Survey of Drug Use and Health (NSDUH) survey information from 2002-2011, nearly 80\% of heroin users reported non-medical prescription pain reliever use prior to their heroin use. Here, non-medical prescription use is defined as taking prescriptions that were not prescribed to the user directly or used only for the feelings it causes. In fact, those who had prior non-medical prescription pain reliever use were 19 times more likely to initiate heroin use than those without prior use \cite{Muhuri}. This could in part be due to the higher availability of heroin in recent years at a lower cost than alternative opioids \cite{NIDA1}. Moreover, approximately 3.6 percent of non-medical prescription pain reliever users began using heroin within 5 years of their first opioid; although a small percentage, this is a significant number of individuals given the magnitude of opioid addiction \cite{Muhuri}. In 2010, an abuse-deterrent formulation of the commonly abused prescription opioid OxyContin was released that made abuse through injection and inhalation more challenging. Although done with the intent of reducing opioid abuse, studies showed that many individuals switched to heroin use instead \cite{Cicero2, Cicero3}. One study of young adults ages 18-25 concluded that among many factors including race, education status, martial status, other drug use, as well as many others, the use of non-prescribed opioid pain relievers in the past year was the biggest indicator of an individual using heroin in the past month, past year or in their lifetime \cite{Ihongbe}. Comparing NSDUH data from 2002-2004 and 2008-2010, the average yearly rates of past year heroin use increased among non-medical opioid users, but heroin use stayed static for those who reported no non-medical use of opioids; the highest rate of heroin use was among individuals with past year non-medical use of opioids ranging between 100 and 365 days of use \cite{Jones}. In the 1960's, heroin users were composed mainly of younger, nonwhite men in urban areas with their initial opioid being heroin, but in recent decades, this trend has shifted to older, white, rural and suburban men and women with their initial opioid being a prescription \cite{Cicero}. Opioids are of no shortage in society today and since opioid addiction is driven largely by legal prescription medication availability, this has made a significant proportion of society susceptible to misuse and addiction to opioids, including heroin. 
\textcolor{blue}{Add more in here from background papers?}\\ 


\textcolor{green}{Goal paragraph} \\ \\
\textcolor{blue}{Goals for our model/questions trying to answer:}
The misuse of opioids, including prescription pain relievers, synthetic opioids, and the illegal drug heroin, is rampant in today's society \cite{NIH2}. The opioid crisis was declared a public health emergency in October 2017 by the United States Department of Health and Human Sciences \cite{HHS1}. It was estimated that the total economic cost of prescription opioid dependence, abuse and overdoses in 2013 alone was \$78.5 billion \cite{Florence}. Due to the health risks of the addicted individuals, public health concerns, the number of overdose deaths and the economic burden, this is an issue worth paying attention to.
To address this apparent problem in today's society, we wish to model the prescription opioid/heroin/fentanyl epidemic in order to understand the dynamics behind the epidemic and predict the trajectory of the epidemic. We also wish to identify important conditions relating to the reduction of opioid/heroin/fentanyl addiction. To do that, we have formulated a population level system of ordinary differential equations model consisting of classes of individuals taking prescription opioids, addicted to opioids, using heroin or fentanyl, and recovering from addiction to opioids, heroin and/or fentanyl, and analyzed it. Our overall goal is to explore how different management strategies may alter the epidemic trajectory; specifically, we would like to investigate management strategies for optimally treating pain with prescriptions while reducing opioid, heroin, and fentanyl addiction.

\textcolor{green}{Other models} \\ \\
\textcolor{blue}{Previous models involving heroin:} \\
There have been several models formulated focusing on heroin addiction. Most of them were motivated by and extensions of the White and Comiskey ordinary differential equation model with three compartments each representing a different stage as a drug-user: the susceptible class including individuals aged 15-64, the drug user class composed of individuals not in treatment, and finally, the drug users in treatment. In this model, individuals in treatment for drug use were only able to die, relapse to drug use, or complete treatment and be immune to drug use for the remainder of the modeling time period. The basic reproduction number, $\mathscr{R}_0$, was calculated and deemed most sensitive to the rate of individuals in the susceptible class becoming drug users; therefore, prevention is more important than treatment for reducing drug use \cite{White}. Wang, Yang and Li did analysis of this model, changing the assumption that the population was not constant \cite{Wang}. 

%\textcolor{blue}{Question: How can the model assume the three compartments make up the entire population, but then have the option for immunity to drug addiction for the remainder of the modeling period? Answer: NOT necessarily the same people; just equal number of people (i.e. someone is born susceptible to replace someone who became immune, maybe)  

The White and Comiskey model was modified later on by Liu and Zhang in order to incorporate a relapse distribution in which there was a non-constant time to relapse. They formulated a delay-differential equation, and also took away the assumption that the population was constant. Their conclusions about the sensitivity of $\mathscr{R}_0$ to certain parameters were in line with those from White and Comiskey \cite{Liu}. Using the same three compartments, another model was formulated with the idea of the relapse rate relying on the length of time the individual has been undergoing treatment. This coupled ODE-PDE model assumed that susceptibles could only become addicted via interaction with heroin users not undergoing treatment. Prevention was deemed more important than treatment, yet again \cite{Fang1}. Fang, Li, Martcheva and Caialso (2015) also modeled the heroin epidemic allowing susceptibility of becoming a drug user to depend on age, again resulting in a coupled ODE-PDE model \cite{Fang2}. A non-autonomous version of the White and Comiskey model included a distributed time delay for becoming a drug user along with both the parameters and the total high-risk population size being time-dependent \cite{Samanta}. From this, an autonomous model was developed incorporating the distributed time delay, which was converted into a discrete model with the time step-size being one \cite{Abdurahman}. Finally, a three compartment model consisting of susceptibles, users not in treatment, and those in treatment, assumed that those not in treatment could not recover and return to being susceptible, and that it took two contacts with a drug user to become addicted, resulting in a nonlinear incidence rate; results showed that perhaps the heroin prevalence undergoes periodic changes depending on conditions \cite{Ma}. 
%\textcolor{blue}{Question: Should I go through each paper specifically, like I have above, or moreso "group them" by stating some ways that the White and Comiskey model was modified and then reference them all together?} Answer: do them individually, that's better 

A deviation from the three compartmental model was done that included a fourth compartment which consisted of individuals who successfully recovered from their heroin use, whether through deciding to stop use by themselves or via treatment measures. The treatment of heroin users is limited by the availability of treatment facilities and resources and therefore, a saturated treatment function is incorporated; the saturation parameter was concluded as having the most vital role in the persistence of the epidemic. This implies that intervention is important early on before heroin addicts accumulate in the community and thus, prompt treatment is necessary \cite{Wangari}. 
 %http://journals.plos.org/plosone/article/file?id=10.1371/journal.pone.0102263&type=printable) 
Another model consisting of 84 ODE's captured the dynamics between prescription opioid users with acute pain, those with chronic pain, illicit opioid users, heroin users and those who overdose. Results showed that increasing addiction treatment and reducing prescriptions for individuals with chronic pain seemed to have a greater impact on reducing both heroin overdose deaths and the number of opioid abusers compared to reducing prescriptions for acute pain without treatment increases \cite{Benneyan}. \textcolor{red}{This was from a conference preceding...in future, look for published model from Benneyan.} 

%\textcolor{blue}{Question: Should only include population level models, correct...not something like a heroin market model (agent-based) with drug users, sellers, homeless \& police?} Will put in that model, and can do anything related because if those authors are editors, they will not like if they don't see their own work referenced. 

Others have taken a completely different approach, specifically using agent-based modeling. In addition to their ODE model, Benneyan, Garrahan, Ilie\cb{s}, and Duan (2017) formulated two agent-based models: the first being a cellular automata model consisting of a grid with neighboring cells updated depending on state variables adjacent to the cells; the second being a network model with nodes that represent individuals/subpopulations and weighted pathways dependent on relational strength and nearness. \textcolor{red}{Not all variables are defined, such as P(x,y), so should I ask for full model from Benneyan?} The former was to better understand regional spread and the latter to investigate spread among individuals; their results suggested that with various regulation practices for opioids, heroin death rates would decrease, but the authors recognize the reality of unintended consequences of these legislations \cite{Benneyan}. Agar (2001) created an agent-based model, consisting of a world with 2,500 patches and 100 agents each assigned an experimentation value which defines how likely that individual is to experiment with heroin, depending on their interest in trying something new, their attitude toward illicit drug use, and their desire to follow societal rules. The model was able to track the number of heroin experimenting individuals over time with the agent moving randomly, one patch selected at random to contain heroin in the beginning, and the ability for individuals to obtain heroin from their original patch, from others who have experimented, or from another patch where the transferring of heroin between agents occurred \cite{Agar}.\textcolor{red}{Put in brief explanation of ``Reducing the complexity of an agent-based local heroin market model" paper by Heard. Look up if any models out there regarding fentanyl use/addiction.} \\

\textcolor{blue}{Summary of Christopher's opioid paper and why formulating another model:} \\
\textcolor{red}{Update with new revisions/any new results from version that was accepted!!} The motivation for our heroin model came from a previous model focusing on opioid addicts through prescriptions or via the black market \cite{Battista}. In order for the model to have an addiction-free equilibrium, both addictions that come from prescriptions and addictions from accessibility to excess prescription drugs must be eliminated, which equates to the need for close administration and monitoring of those prescribed. Furthermore, near the addiction-free state, the prevention of prescription opioid users becoming addicted is more important for staying near the addiction-free equilibrium than reducing the number of prescriptions getting into the hands of non-prescribed users. However, away from the addiction-free state, a realistic situation, the most important factors in reducing the number of addicted individuals include increasing the prescription completion rate, increasing entry into treatment (even among low success rates), and decreasing the prescription rate. The purpose of formulating a separate model from this prescription opioid model is to be able to understand the more complicated dynamics that arise among opioid addiction with the addition of heroin/fentanyl use. As exemplified previously, heroin and fentanyl play a significant role in the process of opioid addiction and recovery and so it's inclusion provides a more accurate overall picture of the epidemic. In addition, since heroin users overdose almost as much as prescription opioid users and in recent years, a sharp increase of heroin overdoses has occurred primarily due to the involvement of fentanyl, it is crucial to take a more detailed look at this increasing problem \cite{CDC4}. In addition, we wish to alter how the recovery process is viewed, by thinking about those in recovery still as addicts, and not having the ability to become a susceptible individual again. 




\textcolor{blue}{How our model is different than previous models} \\
\textcolor{red}{Include details on how my model different than the previous ones.} For the most part, models have not incorporated the connection between prescription opioid misuse and heroin or fentanyl use at all, instead focusing solely on heroin addiction and recovery. Although there are many factors that play a role in population size, susceptibility to drug use, relapse time and other parts of the drug-using process, we consider individuals in each of the classes to be homogeneous, for simplicity of a starting model. \textcolor{blue}{Can I say: discussion of how our model differs from the prescription opioid model (Christopher's) will be discussed after the description of the model to be able to give more details?}


\textbf{Model Formulation} \\ \\
\textbf{Analysis of the Model} \\ \\
\textcolor{red}{Fill in updated addiction-free equilibrium work.} \\
\textcolor{red}{Update basic reproduction number from new addiction-free equilibrium.} \\ \\
\textbf{Numerical Results} 
(literature information and parameter estimation) \\ \\
\textbf{Conclusions} \\ \\
 \textcolor{red}{What did we show that we didn't know before}



\pagebreak

\bibliography{HeroinModel}

 \end{document}

