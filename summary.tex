\documentclass[11pt]{report}
\usepackage{geometry}[1in]
\usepackage{amsmath}

\begin{document}
\begin{center}
{\Huge Heroin Model Summary}\\	
\end{center}
\vspace{1in}

\subsection*{Original}

\begin{align*}
	\frac{dS}{dt} &= \beta E - \alpha S - \underbrace{S(\sigma + \xi(I+H))}_{\text{to resistant}} - \underbrace{S(\delta_1 E + \delta_2 I)}_{\text{to opioid addiction}} - \underbrace{S(\mu_1+\mu_2H)}_{\text{to heroin addiction}}\\
	&\ \ \ \ + \underbrace{rR}_{\text{resist loss}} - gd_SS + (1-g)(d_S(E+R) + d_II + d_HH)\\
	\frac{dE}{dt} &= - \beta E + \alpha S - \underbrace{E(\sigma + \xi(I+H))}_{\text{to resistant}} - \underbrace{E(\mu_1 + \mu_2H)}_{\text{to heroin addiction}} - \underbrace{\epsilon E}_{\text{to I}} - d_S E\\
	\frac{dI}{dt} &= \underbrace{\epsilon E}_{\text{from E}} + \underbrace{S(\delta_1E+\delta_2I)}_{\text{from S}} - \underbrace{\gamma(H)I}_{\text{to H}} - \underbrace{iI}_{\text{to R}} - d_II\\
	\frac{dH}{dt} &= \underbrace{\gamma(H)I}_{\text{from I}} + \underbrace{(S+E)(\mu_1+\mu_2H)}_{\text{from S and E}} - \underbrace{hH}_{\text{to R}} - d_HH\\
	\frac{dR}{dt} &= \underbrace{(S+E)(\sigma + \xi(I+H))}_{\text{from S and E}} + \underbrace{iI}_{\text{from I}} + \underbrace{hH}_{\text{from H}} - 
	\underbrace{rR}_{\text{resist loss}} - (1-g)d_SR\\ 
	&\ \ \ \ + g(d_S(S+E) + d_II + d_HH)\\
	\gamma(H) &= \gamma_0(\mu_1+\mu_2H)
\end{align*}

We assume that $0\leq\delta_1\leq\delta_2\leq 1$, $\mu_1<\mu_2$, and $d_r$ includes both natural death and backsliding from drug resistant to susceptible. If $g=0$ you get back the June version of the discrete-time dynamical system.

\paragraph{Results at equilibrium using the default parameters} put the resistant population quite high, at around 0.77 of the total population. Susceptibles are around 0.2, and the total number of opioid and heroin addicts are 6,827,303 and 410,016 respectively. The opioid number is likely a bit high and the heroin number low, since according to the American Society of Addiction Medicine, of the 20.5 million Americans 12 or older that had a substance use disorder in 2015, 2 million had a substance use disorder involving prescription pain relievers and 591,000 had a substance use disorder involving heroin.

\subsection*{I and H feeds S}

\begin{align*}
\frac{dS}{dt} &= \beta E - \alpha S - \underbrace{S(\sigma + \xi(I+H))}_{\text{to resistant}} - \underbrace{S(\delta_1 E + \delta_2 I)}_{\text{to opioid addiction}} - \underbrace{S(\mu_1+\mu_2H)}_{\text{to heroin addiction}}\\
&\ \ \ \ + \underbrace{iI}_{\text{from I}} + \underbrace{hH}_{\text{from H}} 
+ \underbrace{rR}_{\text{resist loss}} - gd_SS + (1-g)(d_S(E+R) + d_II + d_HH)\\
\frac{dE}{dt} &= - \beta E + \alpha S - \underbrace{E(\sigma + \xi(I+H))}_{\text{to resistant}} - \underbrace{E(\mu_1 + \mu_2H)}_{\text{to heroin addiction}} - \underbrace{\epsilon E}_{\text{to I}} - d_S E\\
\frac{dI}{dt} &= \underbrace{\epsilon E}_{\text{from E}} + \underbrace{S(\delta_1E+\delta_2I)}_{\text{from S}} - \underbrace{\gamma(H)I}_{\text{to H}}\ \ 
- \underbrace{(i+d_I)I}_{\text{to S, rec. \& death}}\\
\frac{dH}{dt} &= \underbrace{\gamma(H)I}_{\text{from I}} + \underbrace{(S+E)(\mu_1+\mu_2H)}_{\text{from S and E}}\ \ - \underbrace{(h+d_H)H}_{\text{to S, rec. \& death}}\\
\frac{dR}{dt} &= \underbrace{(S+E)(\sigma + \xi(I+H))}_{\text{from S and E}} -
\underbrace{rR}_{\text{resist loss}} - (1-g)d_SR\\ 
&\ \ \ \ + g(d_S(S+E) + d_II + d_HH)\\
\gamma(H) &= \gamma_0(\mu_1+\mu_2H)
\end{align*}

\paragraph{Results at equilibrium using the default parameters} lower the number of resistant people considerably, to 0.45 of the total population. Susceptibles rise in a corresponding manner, to 0.4. There are significantly more opioid addicts (21,084,408) and more heroin addicts (1,585,819). 

\subsection*{No R}

\begin{align*}
\frac{dS}{dt} &= \beta E - \alpha S - \underbrace{S(\delta_1 E + \delta_2 I)}_{\text{to opioid addiction}} - \underbrace{S(\mu_1+\mu_2H)}_{\text{to heroin addiction}} 
+ \underbrace{iI}_{\text{from I}} + \underbrace{hH}_{\text{from H}} 
+ d_I I + d_H H + d_S E\\
\frac{dE}{dt} &= - \beta E + \alpha S - \underbrace{E(\mu_1 + \mu_2H)}_{\text{to heroin addiction}} - \underbrace{\epsilon E}_{\text{to I}} - d_S E\\
\frac{dI}{dt} &= \underbrace{\epsilon E}_{\text{from E}} + \underbrace{S(\delta_1E+\delta_2I)}_{\text{from S}} - \underbrace{\gamma(H)I}_{\text{to H}}\ \ 
- \underbrace{(i+d_I)I}_{\text{to S, rec. \& death}}\\
\frac{dH}{dt} &= \underbrace{\gamma(H)I}_{\text{from I}} + \underbrace{(S+E)(\mu_1+\mu_2H)}_{\text{from S and E}}\ \ - \underbrace{(h+d_H)H}_{\text{to S, rec. \& death}}\\
\gamma(H) &= \gamma_0(\mu_1+\mu_2H)
\end{align*}

\paragraph{Leaving off a resistant population} does not force everyone to heroin. As might be expected, the numbers of both opioid addicts and heroin addicts do increase (37,428,464 and 4,344,805), but this could likely be controlled for using different parameter values.

\end{document}
