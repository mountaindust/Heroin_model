\documentclass[12pt]{article}

\begin{document}

\noindent \textbf{Dr. Sharon Davis, DNP (UTK Clinical Assistant Professor in the College of Nursing), meeting notes from 11/19/18:} \\ \\
-Mentored by Dr, Steve Loyd, Director of Tennessee Substance Abuse Services, a recovering addict himself. She has a Health Resources and Services Administration HRSA) grant for opioids. \\

\noindent \textbf{Questions answered:} \\
1) What influences doctors most regarding prescribing behavior: law and professional organizations \\
2) Main concern for doctors: first, decreasing total number of opioid prescriptions (want to reduce number of pills in the state), then decrease the number of opioid patients \\
-CSMD sends letters to top 50 places that prescribe the most in TN each year, usually pain clinics, and no one wants to be on that list \\
-Used to be that anyone could open a pain clinic as long as had medical director (who may know nothing about opioid addiction) but Knox County Pain Coalition looks at legislation and reduced number of pain clinics in the county from 38 to 24. In August 2015, there were 304 pain clinics in Tennessee, and now there are 153. \\
-Now have to show driver's license to pick up prescriptions. \\
3) How medical professionals get on the path of prescribing too much: in past, pharmaceutical reps such as from Purdue Pharma pushed OxyContin and people were clueless about the effects of these drugs, 5th vital sign pushed through pharmaceutical companies through Journal of Clinical Oncology (no longer since 2016), now pharmaceutical reps must educate as well as market \\
-Insurance reimbursing doctors less and less so prescribing is quick \\
-Hospitals get paid by Medicare based on satisfaction surveys so if addicts go into the ER and don't get the pills they want, they will give a bad survey and that will hurt the hospital. Now doctors must do 2 hours of continuing education on substance abuse. TN Dept of Health now has chronic pain guidelines. Education comes from the top (ANA), but doctors mostly follow TN Board of Nursing and TNA guidelines. \\


\noindent \textbf{Other information:} \\
-Dissertation work was an evidence-based projected, focused on motivational interviewing: counseling technique, focused on giving respect to the addicted individual, hones in on self-motivation and the reasons why the patient wants to change; takes about 17 years usually for evidence-based practices to be implemented. \\
-Tries to educate that addiction is a \textit{chronic brain disease}, and change from discussions to education: particularly implemented in undergraduate curriculum regarding what the risk factors/red flags are for people who divert pills. Teaches course in Population Health. ``Science of addiction" lecture. TN Together--12 competencies from the state for students.  \\
-Physician's assistants, physicians and nurse practitioners can all write opioid prescriptions. \\
- 3 factors for addiction: genetics, environmental, accessibility (ACE scores: the higher the score, the higher the risk for addiction, Ratpark study showed that if rats had many other choices than water+morphine mix, they would go with just water. ``Opposite of addiction is isolation." 80\% of addicted women come from abuse). \\
-Sees a rise in hepatitis A coming up. \\
-Book recommendation: Dream Land by Sam Quinones \\
-``Illegal prescribing" defined as: not for legitimate medical purpose, outside accepted medical practice. \\
-National summit for opioid and heroin in Atlanta each April   \\




\end{document}