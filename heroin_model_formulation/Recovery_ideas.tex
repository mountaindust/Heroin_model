\documentclass[12pt]{article}

\textwidth = 16 cm
\textheight = 24 cm
\oddsidemargin = 0.0 cm
\evensidemargin = 0.0 cm
\topmargin = -2 cm
\parskip = 0.2in
\parindent = 0.0in

\usepackage[numbers]{natbib}
\usepackage{graphicx} 
\usepackage{amssymb, amsmath, amsthm} 
\usepackage{fontenc} 
\usepackage{amscd,latexsym,amsfonts,amstext,amsbsy}
\usepackage{euscript} 
\usepackage{enumerate} 
\usepackage{color}  
\usepackage{physics}
\usepackage[latin1]{inputenc}
\usepackage{tikz}
\usepackage{mathrsfs}
\usetikzlibrary{shapes,arrows}
\usepackage{multicol}
\usepackage{comment}
\usepackage{color,soul}
\usepackage{combelow}
\bibliographystyle{apa}


\begin{document}
\noindent \textbf{Ideas about recovery class:} \\ 

\textit{NEW CLASS DEFINITIONS: The opioid addict and heroin addict classes will consist of those who are actively addicted, are in treatment, or are within 4 weeks post-treatment for opioid or heroin/fentanyl addiction. The recovery class (recovered individuals) will consist of those who did not relapse within 4 weeks post-treatment for opioid or heroin/fentanyl addiction and those who are considered at a stable/successful state of being recovered. Therefore, make assumption that those in the recovery class are not considered addicted.}\\

-91\% of opioid addicts in recovery relapse back to addiction within 8 weeks and 70\% of heroin addicts relapse within 4 weeks FILL IN SOURCES (from opioid paper) \cite{NIH4, SAMSHA4} \\
\textcolor{red}{-We are going to use 4 weeks after treatment as the mark of when people are ``successfully recovered" and can move to R if haven't relapsed (think about how that works with parameters), need to adjust that estimate for opioid addict statistic above. \\
-Can use multiple sources for statistics to deduct information: short-term studies give inform on entering R, and then longer-term studies such as in first year or within three years, some rate relapse, give info on leaving R to go to A or H. May be able to use relapse graph to understand shape and be better able to inform these rates. \\
-If can get rate that people enter R from H for example from parameter estimation (i.e. nuH), and then from data know that 90\% relapse within one month, then do 0.1nuH (i.e. estimate parameter and THEN adjust based on data). Think about because would need to consider the number that are actually in recovery (make an assumption about that, such as 1 in 10 are in recovery from BlueCross BlueShield stat) because otherwise, homogeneously mixed addicts and moving too many to R. (Two levels: how many in active recovery in A and then how many don't relapse in 4 weeks....that's how many go to R). \\
-Main idea: pick something and go with it for now, but just have the details straightened out as far as treatment, after treatment period, and relapse, etc.}

\textbf{Where this came from:} \\
-We have data on the number of ``addicted" individuals in 2015 for both opioids and heroin/fentanyl, but  this number would include individuals who are in recovery \\
-Originally, the recovery class consisted of two very different types of people: those in short-term treatment who have a high chance of relapsing (essentially are still addicted) and those who successfully recovered and not addicted anymore according to our definition of addiction; we don't have data on the number of individuals in our recovery class since it includes those who are in active recovery AND those who have finished treatment successfully for their addiction \\
-Most individuals go into short-term recovery (3-6 weeks long), and since 91\% of opioid addicts in recovery relapse back to addiction within 8 weeks and 70\% of heroin addicts relapse within 4 weeks, we may keep these individuals in the addiction class since they have not fully ``recovered," i.e. at a point where they are less likely to fall back into addiction. \cite{NIH4, SAMSHA4} \\
-Making this change would allow us to have the recovery class be composed of individuals who have been addicted in the past and have finished treatment/are in a longer-term treatment program but not considered actively addicted anymore and should be dealt with differently than susceptibles and than those in short-term treatment; could use data for the number of addicts being those just in A. \\ \\


%2. \noindent OR just keep the recovery class the way it is and just make assumption that those in recovery class are not considered addicted. \\

%3. \noindent OR do 48,000-R(2015), but this would take out too many because this would also take out the number of individuals who are not addicted and have recovered from addiction. \\

%4. \noindent OR from BlueCross BlueShield, 1 in every 10 Tennessean who needs substance abuse treatment receives it, and take out 4,800 from those in opioid addict class and 1,400 from those in heroin class and consider those in recovery. BUT our recovery class also contains individuals who have successfully finished treatment and are not currently addicted according to our definition of addiction, so this number would be too low.   \\ %https://bettertennessee.com/health-brief-addiction/

\pagebreak

\bibliography{HeroinModel}

\end{document}