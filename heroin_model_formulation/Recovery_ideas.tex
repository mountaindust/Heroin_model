\documentclass[12pt]{article}

\textwidth = 16 cm
\textheight = 24 cm
\oddsidemargin = 0.0 cm
\evensidemargin = 0.0 cm
\topmargin = -2 cm
\parskip = 0.2in
\parindent = 0.0in

\usepackage[numbers]{natbib}
\usepackage{graphicx} 
\usepackage{amssymb, amsmath, amsthm} 
\usepackage{fontenc} 
\usepackage{amscd,latexsym,amsfonts,amstext,amsbsy}
\usepackage{euscript} 
\usepackage{enumerate} 
\usepackage{color}  
\usepackage{physics}
\usepackage[latin1]{inputenc}
\usepackage{tikz}
\usepackage{mathrsfs}
\usetikzlibrary{shapes,arrows}
\usepackage{multicol}
\usepackage{comment}
\usepackage{color,soul}
\usepackage{combelow}
\bibliographystyle{apa}


\begin{document}
\noindent \textbf{Ideas about recovery class:} \\ 

1.  \noindent We have data on the number of ``addicted" individuals in 2015 for both opioids and heroin/fentanyl, but I believe this number would include individuals who are in recovery \\
-Right now, the recovery class consists of two very different types of people: those who are in short-term treatment who have a high chance of relapsing (essentially are still addicted) and those who have successfully recovered and are not addicted anymore according to our definition of addiction \\
-We don't have data on the number of individuals in our recovery class since it includes those who are in active recovery AND those who have finished treatment successfully for their addiction \\
-Most individuals go into short-term recovery (3-6 weeks long), and since 91\% of opioid addicts in recovery relapse back to addiction within 8 weeks and 70\% of heroin addicts relapse within 4 weeks, we may keep these individuals in the addiction class since they have not fully ``recovered," i.e. at a point where they are less likely to fall back into addiction. \cite{SAMSHA4, NIH4} \\
-Thus, the recovery class would be for: individuals who continue treatment after 8 weeks of opioid treatment or 4 weeks of heroin treatment or have successfully finished treatment. \\
-Making this change would allow us to have the recovery class be composed of individuals who have been addicted in the past and have finished treatment/are in a longer-term treatment program but not considered actively addicted anymore and should be dealt with differently than susceptibles and than those in short-term treatment; could use data for the number of addicts being those just in A. \\ \\
\textit{MAIN IDEA: make the recovery class those who have made it past 8 weeks for opioid treatment and 4 weeks for heroin treatment and just make assumption that those in the recovery class are not considered addicted.}\\

2. \noindent OR just keep the recovery class the way it is and just make assumption that those in recovery class are not considered addicted. \\

3. \noindent OR do 48,000-R(2015), but this would take out too many because this would also take out the number of individuals who are not addicted and have recovered from addiction. \\

4. \noindent OR from BlueCross BlueShield, 1 in every 10 Tennessean who needs substance abuse treatment receives it, and take out 4,800 from those in opioid addict class and 1,400 from those in heroin class and consider those in recovery. BUT our recovery class also contains individuals who have successfully finished treatment and are not currently addicted according to our definition of addiction, so this number would be too low.   \\ %https://bettertennessee.com/health-brief-addiction/

\bibliography{HeroinModel}

\end{document}