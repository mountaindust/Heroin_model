\documentclass[12pt]{article}


\usepackage[numbers]{natbib}
\usepackage{graphicx} 
\usepackage{amssymb, amsmath, amsthm} 
\usepackage{fontenc} 
\usepackage{amscd,latexsym,amsfonts,amstext,amsbsy}
\usepackage{euscript} 
\usepackage{enumerate} 
\usepackage{color}  
\usepackage{physics}
\usepackage[latin1]{inputenc}
\usepackage{tikz}
\usepackage{mathrsfs}
\usetikzlibrary{shapes,arrows}
\usepackage{multicol}
\usepackage{comment}
\usepackage{color,soul}
\usepackage{combelow}
\bibliographystyle{apa}
\definecolor{applegreen}{rgb}{0.55, 0.71, 0.0}


\textwidth = 16 cm
\textheight = 24 cm
\oddsidemargin = 0.0 cm
\evensidemargin = 0.0 cm
\topmargin = -2 cm
\parskip = 0.2in
\parindent = 0.0in


\newtheorem{theorem}{Theorem}
\newtheorem{problem}[theorem]{Problem}
\newtheorem{exercise}[theorem]{Exercise}
\newtheorem{corollary}[theorem]{Corollary}
\newtheorem{lemma}[theorem]{Lemma}
\newtheorem{proposition}[theorem]{Proposition}
\newtheorem{proporties}[theorem]{Proporties}
\newtheorem{definition}[theorem]{Definition}
\newtheorem{definitions}[theorem]{Definitions}
\newtheorem{example}[theorem]{Example}
\newtheorem{remark}[theorem]{Remark}

 


\begin{document}

\textbf{Data for Heroin Model} \\
Tricia Phillips \\
\today \\
\begin{itemize}

\item We will run our model from 2013-2017, and initial conditions for the classes will be estimated, as we do not have data for the number of individuals in each class at the beginning of 2013.  \\

\item The following are the \textbf{number of individuals in each class} for Tennessee in 2015: \\

*\textbf{Susceptibles}: (Total population in 2015-size of four other classes$=6,590,726-1,819,581-48,000-14,000-R_{0}=$FILL IN) \cite{USCensus} \\

*\textbf{Prescription opioid users}, 2015: 1,819,581 \cite{TNgov1} \\
Although this number does not explicitly state it is for individuals 12 and older, we assume it is since it comes from the Tennessee Department of Health; if it does include individuals under 12, we assume that number is negligible. \\

*\textbf{Opioid addicts}, 2015/2016 average for ``Pain Reliever Use Disorder" for individuals 12 and older: 48,000 \cite{NSDUH2} \\
We note that their definition of pain reliever use disorder includes those who meet the American Psychiatric Association criteria for dependence or abuse. Here, opioid dependence is classified as having "signs and symptoms that reflect compulsive, prolonged self-administration of opioid substances that are used for no legitimate medical purpose or...are used in doses that are greatly in excess of the amount needed for pain relief...regular patterns of compulsive drug use that daily activities are typically planned around obtaining and administering opioids." This definition falls under our definition of opioid addiction. Opioid abuse, on the other hand, they consider to be less severe than dependence, and would not lead to the development of withdrawal symptoms. This latter definition does not fall under our characterization of addiction, but we make a note of this to say that this estimate for those with a pain reliever use disorder may be overestimated for what we are concerned with, but is an acceptable approximation  \cite{DSM}.\\

*\textbf{Heroin/fentanyl addicts}, 2015/2016 average for ``Past Year Heroin Use" for individuals 12 and older: 14,000 \cite{NSDUH2}  \\ 
Although this number includes those who may have used heroin once or twice in the past year, we are under the assumption that the majority of these individuals are addicts and that very few, if any, individuals use heroin recreationally. In addition, the number of heroin users does not include fentanyl users explicitly, but we are under the assumption that those who take fentanyl are a subset of those who use heroin, and therefore, would mostly be included in these numbers. We admit the values may be slightly too low, for the cases of individuals who do fentanyl and not heroin, but data has not been found for fentanyl addicts only. Therefore, we are working under the assumption that it would be a negligible population that takes fentanyl without heroin. Overall, these two assumptions may work to balance one another out. 


%NSDUH Info for how do surveys for states: https://www.samhsa.gov/data/sites/default/files/NSDUHsaeMethodology2016/NSDUHsaeMethodology2016.pdf

\textit{*\textbf{Recovering addicts}: won't be able to find because we do not know the total of individuals total that have been in treatment ever in the past for our time frame}  \\



\item For Tennessee, the \textbf{total number of individuals taking prescription opioids} for pain \cite{TNgov1}: \\
2013: 1,845,144 \\
2014: 1,824,342 \\
2015: 1,819,581 \\
2016: 1,761,363 \\
2017: 1,636,374 

Although this number does not explicitly state it is for individuals 12 and older, we assume it is since it comes from the Tennessee Department of Health; if it does include individuals under 12, we assume that number is negligible. \\

\item For Tennessee individuals 12 and older, \textbf{the number of treatment admissions for non-heroin opiates/synthetics} as the primary substance of abuse to facilities that receive state/public funding (generally referring to funding by the state substance abuse agency) \cite{TEDS2015_SAMSHA_admissions}: \\
2005: 1,578 \\
2006: 1,529 \\
2007: 1,743 \\
2008: 2,022 \\
2009: 2,464 \\
2010: 3,384 \\
2011: 3,884 \\
2012: 4,203 \\
2013: 4,485 \\
2014: 4,530 \\
2015: 4,326 \\

We are under the assumption that if one were addicted to heroin in addition to prescription opioids, their heroin problem would be the primary reason for going to treatment and would be included in the following numbers. 

%can get "new arrivals to R" from this info, apparently: integrate dR/dt over time frame interested in (t_1 to t_2) and on RHS, integrate \zeta*A+\nu*H (or can break into two integrals on the RHS), and can get new arrivals; don't have to worry about double counting or people cycling \\

\item For Tennessee individuals 12 and older, \textbf{the number of treatment admissions for heroin} as the primary substance of abuse to facilities that receive state/public funding (generally referring to funding by the state substance abuse agency) \cite{TEDS2015_SAMSHA_admissions}: \\
2010: 199\\
2011: 240 \\
2012: 390 \\
2013: 555 \\
2014: 743 \\
2015: 1,083 \\


Again, these numbers do not include fentanyl users explicitly, but we are under the assumption that those who take fentanyl are a subset of those who use heroin, and therefore, would mostly be included in these numbers. We admit the values may be slightly too low, for the cases of individuals who do go to treatment with the primary substance of abuse being fentanyl, but there is not data available for those numbers currently. 


%All TEDS tables: https://wwwdasis.samhsa.gov/dasis2/teds.htm

\item We use data on the number of prescription opioid overdose deaths which include natural, semi-synthetic, and synthetic opioids; however, we subtract out the number of fentanyl overdoses (fentanyl is classified as a synthetic prescription opioid), since those overdoses are counted for in their own category, listed below. This results in the following  \textbf{total number of prescription opioid overdose deaths} \cite{PDO}: \\
2013:(637-53=) 584 \\
2014: (697-69=) 628 \\
2015: (848-169=) 679 \\
2016: (1,009-294=) 715 \\

Although this number does not explicitly state it is for individuals 12 and older, we assume it is since it comes from the Tennessee Department of Health; if it does include individuals under 12, we assume that number is negligible. \\
%SEE WORK on 11/19/18 meeting notes for mu_A, mu_H, mu
We may calculate the overdose death rate in the year 2015, since we know the number of addicted individuals in that year. To find the continuous-time rate at which individuals are dying from the addicted class, we consider the equation $k A_{0}=A_{0}e^{-\mu_{A}t}$, where $A_0$ is the number of individuals addicted to opioids in 2015 and $k$ is the proportion of these individuals in the addicted class at the start of 2016 (when $t=1$). In 2015, there were 679 individuals out of the entire Tennessee population 12 and older that overdosed on prescription opioids \cite{PDO}. However, it is estimated that only 54.6\% of these individuals were actually at an increased risk for an opioid-related overdose death; we will assume that if an individual met the criteria for at least one high-risk factor, that they were considered addicted to opioids \cite{Gwira}. Therefore, we will assume 54.6\% of the 679 individuals that overdosed were addicted. With a total of 48,000 opioid addicts in 2015, this means that (48,000-0.546$\cdot$679)/48,000 $\approx$ 0.992 is the proportion of addicted individuals that remain by the beginning of the next year. This implies 0.992$A_0=A_0 e^{-\mu_{A}(1)}$, and solving results in $\mu_{A} \approx 0.00775.$

Similarly, we may calculate $\mu_{H}$. There were 14,000 heroin/fentanyl addicts in 2015 and 374 heroin-related overdoses. We make the assumption that if an individual died of a heroin overdose they were addicted in line with our previous assumptions that if one is using heroin, they are considered addicted due to the nature of the drug. This means that (14,000-374)/14,000 $\approx$ 0.973 is the proportion of heroin users that remain at $t=1$, which implies 0.973$H_0=H_0 e^{-\mu_{H}(1)}$, which results in $\mu_{H}$ $\approx 0.0271$, the continuous-time rate at which individuals are dying from the heroin class.

%we view it as a good thing that methadone was pulled out separately from this data for overdoses, because it's used in treatment in order to reduce cravings and not necessarily supply the high 

\item We add together the heroin and fentanyl overdoses from the years 2013-2016 for the state of Tennessee. The \textbf{total number of heroin and fentanyl overdoses} for these four years are: \\
2013: (63+53=) 116 \\
2014: (147+69=) 216 \\
2015: (205+169=) 374 \\
2016: (260+294=) 554 \cite{PDO}. 

Although this number does not explicitly state it is for individuals 12 and older, we assume it is since it comes from the Tennessee Department of Health; if it does include individuals under 12, we assume that number is negligible. \\

We make a note that individuals that do not have an opioid use disorder and die because of an opioid overdose are counted in the ``natural mortality rate." 


%Could use overdose data of heroin compared to number of heroin addicts in order to get a scaling, and then be able to determine number of fentanyl addicts based on the number of fentanyl overdoses. But for now, going with subset argument, that heroin users includes fentanyl users. 


\item Total population estimated in Tennessee each year \cite{USCensus}: \\
2013: 6,490,795 \\
2014: 6,540,007 \\
2015: 6,590,726 \\
2016: 6,649,404 \\
2017: 6,715,984 \\
\textcolor{red}{2018: FILL IN} \\

There was an estimated 1,073,214 individuals Tennessee in 2018 aged 12 and under. To figure out those who are 12 years old, we take approximately 1/8th of the individuals that are in the age group 5-12, which is approximately 83,175 individuals \cite{DOHHS}. Thus, an estimated 990,039 individuals are \emph{under} the age of 12 in Tennessee. Given the last total population estimate for 2017 being 6,715,984 from above, this means that approximately 15\% of the population is under the age of 12. \textcolor{red}{FIX when get updated 2018 total population}. Since we do not see a reason for this percentage to be significantly different from year to year, we assume that this percentage is constant throughout the time period we are looking at. Then, we are able to consider the following \textbf{Tennessee population estimates for individuals 12 and older} in order to align with the rest of the data that is in this age range by taking off 15\% of the above total population estimates.  \\
2013: 5,517,176 \\
2014: 5,559,006 \\
2015: 5,602,117 \\
2016: 5,651,993 \\
2017: 5,708,586 \\

%Could be helpful if need for individuals under 12: https://factfinder.census.gov/faces/tableservices/jsf/pages/productview.xhtml?src=CF 
\item The \textbf{age-adjusted death rate} for Tennessee in 2016 was calculated to be 886.3 out of 100,000 individuals, or approximately 50,094 people out of a total population 12 and older of 5,651,993 \cite{Kaiser}. Subtracting off the number of people who died from a prescription opioid or heroin/fentanyl overdose in 2016 results in 48,825 people who died that year. This implies that (5,651,993-48,825)/5,651,993 $\approx$ 0.991 is the proportion of the population that remains by the beginning of 2017. If we consider $T_0$ to be the total population in 2016, we can find the continuous-time rate at which individuals die naturally from the equation 0.991$T_0$=$T_0e^{-\mu t}$, which results in the natural death rate $\mu \approx 0.0087$.



\item \textbf{Relationship among $\theta_1$, $\theta_2$, and $\theta_3$:} Since we could not find data for values of $\theta_1$, $\theta_2$, or $\theta_3$ for Tennessee, we consider a national study of individuals 12 and older to establish a relationship among these three rates. For a national study consisting of 609,000 participants, ``the recent heroin incidence rate was 19 times higher among those who reported prior non-medical pain reliever (NMPR) use (0.39\%) than among those who did not report NMPR use (0.02\%) \cite{Muhuri}.
Thus, we will extrapolate this information to say that the rate that prescription opioid users and opioid addicts move to heroin use is 19 times greater than the rate at which susceptibles move to heroin use (i.e. $\theta_2 + \theta_3$ $>$ 19$\theta_1$). Although $\theta_2$ is going from P to H and consists of prescription opioid users who do not misuse their prescription, we will make the assumption that those who misuse are much more likely to be the ones to move to heroin use. \\
\pagebreak

\textcolor{red}{\emph{Ignore for now, later tell story about what model tells us regarding this.}} \\
\textbf{Information for moving from recovery to opioid addiction or heroin/fentanyl addiction, $\sigma_A$ and $\sigma_H$:} 

We consider here the number of individuals in Tennessee age 12 and older who dropped out of treatment and assume that dropping out would result in an individual going back into addiction since they did not successfully complete treatment. The number of drop-outs from medication-assisted opioid detox programs and outpatient medication-assisted opioid programs was reported as essentially zero for Tennessee in 2012, 2013, and 2015, and 3 or fewer for 2011 and 2014, which does not seem realistic \cite{TEDS2011_SAMSHA_discharges, TEDS2012_SAMSHA_discharges, TEDS2013_SAMSHA_discharges, TEDS2014_SAMSHA_discharges, TEDS2015_SAMSHA_discharges}. Therefore, to get an estimate for what this value may look like, we took the total number of admissions into all drug treatment programs for each year and calculated the drop-out rates for each of those years: \\
2011: 2,910/13,422=.217  \cite{TEDS2011_SAMSHA_admissions,TEDS2011_SAMSHA_discharges} \\
2012: 3,127/13,525=.231 \cite{TEDS2012_SAMSHA_admissions, TEDS2012_SAMSHA_discharges} \\
2013: 3,273/14,476=.226 \cite{TEDS2013_SAMSHA_admissions, TEDS2013_SAMSHA_discharges} \\
2014: 3,164/14,909=.212  \cite{TEDS2014_SAMSHA_admissions, TEDS2014_SAMSHA_discharges} \\
2015:  3,039/14,916=.204 \cite{TEDS2015_SAMSHA_admissions, TEDS2015_SAMSHA_discharges} \\

We take the average of these drop out rates and extrapolate this to be an estimate for those dropping out of opioid-related therapies: .218. (SEEMS WAY TOO LOW).

%Additional treatment admissions data for Tennessee: https://wwwdasis.samhsa.gov/dasis2/teds_pubs/2015_teds_rpt_st.pdf


\end{itemize}

\pagebreak

\bibliography{HeroinModel}

 \end{document}