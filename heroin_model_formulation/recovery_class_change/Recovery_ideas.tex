\documentclass[12pt]{article}

\textwidth = 16 cm
\textheight = 24 cm
\oddsidemargin = 0.0 cm
\evensidemargin = 0.0 cm
\topmargin = -2 cm
\parskip = 0.2in
\parindent = 0.0in

\usepackage[numbers]{natbib}
\usepackage{graphicx} 
\usepackage{amssymb, amsmath, amsthm} 
\usepackage{fontenc} 
\usepackage{amscd,latexsym,amsfonts,amstext,amsbsy}
\usepackage{euscript} 
\usepackage{enumerate} 
\usepackage{color}  
\usepackage{physics}
\usepackage[latin1]{inputenc}
\usepackage{tikz}
\usepackage{mathrsfs}
\usetikzlibrary{shapes,arrows}
\usepackage{multicol}
\usepackage{comment}
\usepackage{color,soul}
\usepackage{combelow}
\bibliographystyle{apa}


\begin{document}
\noindent \textbf{Ideas about recovery class:} \\ 
\today \\ \\
\textit{NEW CLASS DEFINITIONS: The opioid addict and heroin addict classes will consist of those who are actively addicted, are in treatment, or are within 4 weeks post-treatment for opioid or heroin/fentanyl addiction. The recovered class will consist of those who completed treatment for opioid or heroin/fentanyl addiction and did not relapse within 4 weeks post-treatment, and therefore are considered in a stable/successful state of being recovered. Thus, we will make the assumption that those in the recovery class are not considered addicted.}\\

\textcolor{blue}{Support/data/arguments for why choosing 4 weeks/acute withdrawal period} \\
We choose to use 4 weeks after treatment with no relapse as the mark of when people are ``successfully recovered" and can move to R. We chose this time frame due to several studies exhibiting a high level of relapse within weeks of discharge from treatment, which suggests individuals are not stably recovered. One study following up with opiate dependent individuals (of which 88\% of the study population were heroin users) showed that 59\% of the individuals discharged from treatment relapsed within 1 week and 71\% within 4 weeks \cite{Smyth}. Moreover, 80\% of individuals who relapsed after treatment did so within 4 weeks and that 92\% of individuals who relapsed after discharge had gone back to treatment prior to their follow-up interview (which was between 18-42 months). These statistics convey the unstable nature of an addicts' recovery, particularly within a few weeks of post-treatment (since that's when a majority of them relapsed). In another opiate dependent group study evaluating their relapse risk, 27\% had relapsed on the discharge day of their most recent treatment program, 41\% had relapsed within 1 week and 65\% within 4 weeks \cite{Bailey}. The only study found specifically for prescription opioid addicts suggested that 91\% of them who completed a two-phase treatment program relapsed back into addiction within 8 weeks post-treatment \cite{Weiss}. On the lower side, in another study an estimated 32\% of the 68 opiate addicts interviewed returned to addiction within 4 weeks and 65\% lapsed, which we note has potential to lead to dependence in time; at the 6 month follow-up, 50\% of the 60 individuals interviewed were dependent \cite{Broers}. Another study following up with opiate users post-treatment for 6 months showed that 32\% had lapsed within 1 week post-discharge when interviewed; 71\% had lapsed by the 6th week; and 44\% had returned to the daily use of opiates (considered addicted) at the 2 month mark. Furthermore, the study stated that several individuals who completed the withdrawal portion of the program lapsed while in the second part of treatment (although not necessarily equating to a full relapse back into addiction) which supports our claim that those who entered treatment addicted should be still be considered addicted while in treatment \cite{Gossop1}. Later analysis of this study regarding relapse factors stated that 25\% of those who lapsed mentioned withdrawal symptoms as playing a role in their lapses \cite{Gossop2}. These statistics suggest that opiate-abstinence can be difficult to achieve in this initial time period post-treatment: ``...the period immediately after leaving a residential treatment is of massively high risk: the great majority of lapses occurred with the first few weeks after discharge...after the first four weeks, there were few additional lapses...first few weeks after discharge as a \textit{critical period} in the process of recovery" \cite{Gossop1}. This suggests four weeks can be viewed as an important marker in recovery, and that leaving individuals who bounce back to addiction/frequently relapse should simply remain in their respective addicted classes. Other studies suggest significant relapse rates within weeks and months for heroin addicts and opiate addicts, as well \cite{Hunt}. \\

% If need: \textcolor{red}{Moreover, a study done in Ireland suggests that the majority of individuals in treatment (82\%) after 3 years of the study are still using illicit drugs, which supports our decision to leave those in treatment in their respective addicted class} Also in Potter source. 


\textcolor{blue}{Relapse comes from primary drug and *not* source-driven, argument for form of our transition terms.} \\
\textcolor{red}{Add this information into the description of the terms in journal article document about assumptions making, i.e. this data is about relapsing back to where individuals came from (so we will make that assumption that individuals are going back to their primary drug of choice, so we do that with our proportion terms).} \\
In one study, 63\% of initial lapses occurred around other opioid users, 48\% in another addicts' home, and 31\% in ones own home \cite{Gossop1}. This may lead one to believe that interactions with other addicts would play a large role with an initial lapse/relapse; however, following this study, the authors explored the circumstances that led to the lapses of these study subjects and results proved differently. The top factors that led to these initial lapses were cognitive factors (planned to use), mood states (angry, sad, lonely, etc.), and external influences (situations unrelated to drug use that led to use). This was taking into consideration the number of people who mentioned the factors, the total number of times the factors were mentioned, the number of individuals who deemed the factor most important, and finally, the number of individuals that indicated the factor was the initiator of a sequence of other factors, as lapse often is a result of multiple factors occurring simultaneously. These were also the factors that were primarily contributed to the continued use of opiates. It seems clear that source-driven factors, specifically social pressure (offered drugs), drug availability (in another users' home or in the area), drug-related clues (objects used for drug-taking/observing others under the influence), and interpersonal influences (seeing or thinking about certain people) had relatively little importance compared to the others mentioned above \cite{Gossop2}. It is because of this that we argue individuals return to their primary drug of choice compared to what is simply available; although we are interested specifically in a return to addiction, that is umbrellaed under the broader category of lapses. \\
We mention that another study specifically on heroin users enabled the study subjects to choose categories they deemed as most important in their relapse according to a phrase attached to the category and this was compared to the results of independent judges selecting the categories based on individuals' responses. The highest mean ratings for the categories that the subjects chose were for the giving into temptations/urges in the presence of substance cues and direct social pressure, whereas the judges deemed negative emotional states and indirect social pressure as the most frequent reasons. However, given that subjects were provided a phrase that correlated with a certain category, one could argue that if an individual did not agree with the exact phrase, they may have chosen an alternative option that they related to more (e.g. ``I felt bored" represented the entire category of ``negative emotional states other than frustration and/or anger.). Moreover, the previous study applies both to opioid addicts and heroin addicts, rather than just heroin addicts \cite{Heather}.

\pagebreak

\textcolor{blue}{If end up using data in future for estimating relapse rates:} \\
-We will use multiple sources to deduct information since short-term studies give information on transitioning from A or H to R, and longer-term studies, such as relapse rates within the first year or within three years post-treatment, give information on transitioning from R to A or H. \\
 -Use 91\% in 8 weeks for prescription opioid users and 71\% of heroin users in 4 weeks \cite{Weiss, Smyth}. Although these are on a national level, we will assume the rates do not differ significantly for Tennessee, as we were not able to find any relapse statistics specifically for the state. We were not able to find information on the relapse rate within 4 weeks after treatment for prescription opioid addicts. Therefore, we will assume it is approximately the same as heroin, (see heroin\_journal\_article: using information that suggests high percentage of relapse within a few weeks to justify why 70\% and why assuming approximately same percentage as heroin
 %later could add more about same pharmacological properties \\
-Long-term: information for transitioning from R to A or H:
A few studies showed that 90\% of individuals addicted to opioids or heroin relapsed within one year post-treatment and 91\% for opiate dependent individuals at some point post-treatment (with 88\% of this study population being heroin users) \cite{Bailey, Smyth}. Another study consisting of a two-phase treatment program and follow-up study had 61\% of prescription opioid addicts abstaining from opioid use in the past month at the 3.5 year mark after treatment. \\
%Other source: http://asitrainingonline.net/wp-content/uploads/2012/06/p-Monograph-72.Relapse-and-recovery-in-Drug-Abuse-pdf.pdf#page=95 
%-If we can get rate that people enter R from H from parameter estimation (i.e. nuH), and then from data know that 70\% relapse within one month, then do 0.3nuH to get total that go to R (i.e. estimate parameter and THEN adjust based on data).\\ 
%\textcolor{red}{-But wouldn't this be if the time step were one month? And not one year? \textcolor{blue}{Tricia: think about units of parameters} \\
-Need to consider the number of people that are actually in treatment (1 in 10 from https://addiction.surgeongeneral.gov/key-findings/early-intervention) AND successfully finish treatment because otherwise, homogeneously mixed addicts and moving too many to R? Make assumption that one cannot move to a stable recovered class without treatment of some sort since addiction is a disease? (Three levels: how many in active recovery in A or H AND finish treatment AND don't relapse in 4 weeks afterward....that's how many go to R. For example: 1 in 10 heroin addicts are in recovery, 53.3\% finish treatment overall (from TEDS2015), and then 30\% don't relapse within 4 weeks of treatment, so (.1)(.2)(.3)nuH?) YES.
%\textcolor{red}{I believe ROSIE study says somewhere that individuals in treatment are mostly still taking drugs, which supports our decision to leave those in treatment in an addicted class: https://www.drugsandalcohol.ie/11542/1/ROSIE3-YearReport.pdf} \\ 

\textcolor{red}{-could not find opioid stat for 4 weeks \\
-could not find any sort of relapse rate graph over time to understand shape and be better able to inform the rates \\
-could not find any acute stage withdrawal sources that suggested the distribution of relapsing individuals in the weeks following treatment}  \\
%Overall helpful explanation of sources/ideas relating to recovery: https://books.google.com/books?hl=en&lr=&id=Sj-NZMIF-QcC&oi=fnd&pg=PA1&dq=relapse+to+opioid+dependence&ots=lQSBpU6GY1&sig=XLX9Hj69l5cQMKPd_rB5CodnVBo#v=snippet&q=%20opioid%20&f=false


%-Main idea: pick something and go with it for now, but just have the details straightened out as far as treatment, after treatment period, and relapse, etc.

\textbf{Where this came from:} \\

\textit{MAIN IDEA: The recovery class we initially defined consisted of two very different groups of individuals: those with a high chance of relapse and those with a much lower chance of relapse. Also, data for the number of addicted individuals included those in recovery, which we did not know the number for. So we will redefine the recovery class to a recovered class as described above.} \\

-We have data on the number of ``addicted" individuals in 2015 for both opioids and heroin/fentanyl, but  this number would include individuals who are in recovery \\
-Originally, the recovery class consisted of two very different types of people: those in short-term treatment who have a high chance of relapsing (essentially are still addicted) and those who successfully recovered and not addicted anymore according to our definition of addiction; we don't have data on the number of individuals in our recovery class since it includes those who are in active recovery AND those who have finished treatment successfully for their addiction \\
-Since 91\% of opioid addicts in recovery relapse back to addiction within 8 weeks post-treatment and 70\% of heroin addicts relapse within 4 weeks post-treatment, we wish to keep these individuals in the addiction class since they have not fully ``recovered," i.e. at a point where they are less likely to fall back into addiction. \cite{Weiss, Smyth}\\
-Making this change would allow us to have the recovery class be composed of individuals who have been addicted in the past and have finished treatment but not considered actively addicted anymore and should be dealt with differently than both susceptibles and those in short-term treatment; could use data for the number of addicts being those just in A. \\ 



%2. \noindent OR just keep the recovery class the way it is and just make assumption that those in recovery class are not considered addicted. \\

%3. \noindent OR do 48,000-R(2015), but this would take out too many because this would also take out the number of individuals who are not addicted and have recovered from addiction. \\

%4. \noindent OR from BlueCross BlueShield, 1 in every 10 Tennessean who needs substance abuse treatment receives it, and take out 4,800 from those in opioid addict class and 1,400 from those in heroin class and consider those in recovery. BUT our recovery class also contains individuals who have successfully finished treatment and are not currently addicted according to our definition of addiction, so this number would be too low.   \\ %https://bettertennessee.com/health-brief-addiction/

\pagebreak

\bibliography{HeroinModel}

\end{document}