\documentclass[12pt]{article}

\textwidth = 16 cm
\textheight = 24 cm
\oddsidemargin = 0.0 cm
\evensidemargin = 0.0 cm
\topmargin = -2 cm
\parskip = 0.2in
\parindent = 0.0in

\usepackage[numbers]{natbib}
\usepackage{graphicx} 
\usepackage{amssymb, amsmath, amsthm} 
\usepackage{fontenc} 
\usepackage{amscd,latexsym,amsfonts,amstext,amsbsy}
\usepackage{euscript} 
\usepackage{enumerate} 
\usepackage{color}  
\usepackage{physics}
\usepackage[latin1]{inputenc}
\usepackage{tikz}
\usepackage{mathrsfs}
\usetikzlibrary{shapes,arrows}
\usepackage{multicol}
\usepackage{comment}
\usepackage{color,soul}
\usepackage{combelow}
\bibliographystyle{apa}


\begin{document}
\noindent \textbf{Ideas about recovery class:} \\ 

\textit{NEW CLASS DEFINITIONS: The opioid addict and heroin addict classes will consist of those who are actively addicted, are in treatment, or are within 4 weeks post-treatment for opioid or heroin/fentanyl addiction. The recovered class will consist of those who completed treatment for opioid or heroin/fentanyl addiction and did not relapse within 4 weeks post-treatment, and therefore are considered in a stable/successful state of being recovered. Thus, we will make assumption that those in the recovery class are not considered addicted.}\\



-We will use multiple sources to deduct information since short-term studies give information on entering R from A or H, and longer-term studies such as a relapse rate within first year or within three years post-treatment, give information on leaving R to go to A or H. \\
-We are going to use 4 weeks after treatment with no relapse as the mark of when people are ``successfully recovered" and can move to R. \textcolor{blue}{Tricia: Argue why choosing 4 weeks as timeline: because data suggests high level of relapse in that time frame which suggests not stably recovered and look for sources that talk about acute withdrawal (Gossop 1987) and bouncing back to addiction after treatment/frequent relapse} We chose this time frame because one follow-up study with opiate dependent individuals (of which 88\% of the study population were heroin users) showed that 80\% of individuals who relapsed after treatment did so within one month and that 92\% of individuals who relapsed after discharge had gone back to treatment prior to their follow-up interview, which conveys the more unstable nature of their recovery, particularly within a few weeks of post-treatment (since that's when a majority of them relapsed) \cite{Smyth}. This same study exhibited a 71\% relapse rate within 4 weeks for this group of mostly heroin users \cite{Smyth}. 
Another study suggested that 91\% of prescription opioid addicts who completed a two-phase treatment program relapsed back into addiction within 8 weeks post-treatment \cite{Weiss}. Although these are on a national level, we will assume the rates do not differ significantly for Tennessee, as we were not able to find any relapse statistics specifically for the state. We were not able to find information on the relapse rate within 4 weeks after treatment for prescription opioid addicts. Therefore, we will assume it is approximately the same as heroin, \textcolor{blue}{Tricia: find source that suggests high percentage of relapse within a few weeks to justify why 70\% and why assuming approximately same percentage as heroin. Find source that talks about having similar pharmacological properties (one of the government websites, I think)} One study discusses that 59\% of individuals opiate dependent and discharged from treatment relapsed within one week \cite{Smyth}. In another study group looking at their risk of relapse, 27\% had relapsed on their discharge day of their most recent treatment program, 41\% had relapsed within one week and 65\% within four weeks \cite{Bailey}.

-90\% of individuals addicted to opioids or heroin relapsed within one year post-treatment and 91\% for opiate dependent individuals at some point post-treatment (with 88\% of this study population being heroin users) \cite{Bailey, Smyth}. Another study consisting of a two-phase treatment program and follow-up study had 61\% of prescription opioid addicts abstaining from opioid use in the past month at the 3.5 year mark after treatment. \textcolor{blue}{Tricia: look for other sources, too.}. 

-From here, if we can get rate that people enter R from H from parameter estimation (i.e. nuH), and then from data know that 70\% relapse within one month, then do 0.3nuH to get total that go to R  (i.e. estimate parameter and THEN adjust based on data).\\ 
\textcolor{red}{-But wouldn't this be if the time step were one month? And not one year? \textcolor{blue}{Tricia: think about units of parameters} \\
-Also, do we need to consider the number of people that are actually in recovery (make an assumption about that, such as 1 in 10 are in recovery from BlueCross BlueShield stat) AND successfully finish treatment because otherwise, homogeneously mixed addicts and moving too many to R? Make assumption that one cannot move to a stable recovered class without treatment of some sort since addiction is a disease? (Three levels: how many in active recovery in A or H AND finish treatment AND don't relapse in 4 weeks afterward....that's how many go to R. For example: 1 in 10 heroin addicts are in recovery, 20\% finish treatment overall, and then 30\% don't relapse within 4 weeks of treatment, so (.1)(.2)(.3)nuH?) YES.} \textcolor{blue}{Tricia: look for overall rate of entering treatment and of finishing treatment.\\}

\textcolor{red}{-could not find opioid stat for 4 weeks \\
-could not find any sort of relapse rate graph over time to understand shape and be better able to inform the rates \\
-could not find any acute stage withdrawal sources that suggested the distribution of relapsing individuals in the weeks following treatment}  \\

%-Main idea: pick something and go with it for now, but just have the details straightened out as far as treatment, after treatment period, and relapse, etc.

\textbf{Where this came from:} \\

\textit{MAIN IDEA: The recovery class we initially defined consisted of two very different groups of individuals: those with a high chance of relapse and those with a much lower chance of relapse. Also, data for the number of addicted individuals included those in recovery, which we did not know the number for. So we will redefine the recovery class to a recovered class as described above.} \\

-We have data on the number of ``addicted" individuals in 2015 for both opioids and heroin/fentanyl, but  this number would include individuals who are in recovery \\
-Originally, the recovery class consisted of two very different types of people: those in short-term treatment who have a high chance of relapsing (essentially are still addicted) and those who successfully recovered and not addicted anymore according to our definition of addiction; we don't have data on the number of individuals in our recovery class since it includes those who are in active recovery AND those who have finished treatment successfully for their addiction \\
-Most individuals go into short-term recovery (3-6 weeks long), and since 91\% of opioid addicts in recovery relapse back to addiction within 8 weeks and 70\% of heroin addicts relapse within 4 weeks, we wish to keep these individuals in the addiction class since they have not fully ``recovered," i.e. at a point where they are less likely to fall back into addiction. \cite{NIH4, SAMSHA4, Weiss, Smyth}\\
-Making this change would allow us to have the recovery class be composed of individuals who have been addicted in the past and have finished treatment but not considered actively addicted anymore and should be dealt with differently than both susceptibles and those in short-term treatment; could use data for the number of addicts being those just in A. \\ 



%2. \noindent OR just keep the recovery class the way it is and just make assumption that those in recovery class are not considered addicted. \\

%3. \noindent OR do 48,000-R(2015), but this would take out too many because this would also take out the number of individuals who are not addicted and have recovered from addiction. \\

%4. \noindent OR from BlueCross BlueShield, 1 in every 10 Tennessean who needs substance abuse treatment receives it, and take out 4,800 from those in opioid addict class and 1,400 from those in heroin class and consider those in recovery. BUT our recovery class also contains individuals who have successfully finished treatment and are not currently addicted according to our definition of addiction, so this number would be too low.   \\ %https://bettertennessee.com/health-brief-addiction/

\pagebreak

\bibliography{HeroinModel}

\end{document}